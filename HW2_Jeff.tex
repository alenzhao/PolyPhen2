\documentclass{article}

\usepackage{hyperref}
\usepackage{graphicx}

\begin{document}



\centerline{\sc \large CompSci 590 HW2}
\vspace{1pc}
\centerline{\sc Molecular Assembly and Computation }
\vspace{1pc}

\centerline{\it (Question one through Question seven)}

\vspace{1pc}
\centerline{\sc  Guangjian (Jeff) Du}
\vspace{2pc}

% % % % % % % % % % % % % % % % % % % % % % % % % % %% % % % % % % % % % % % % % % % % % % %
% % % % % % % % % % % % % % % % % % % % % % % % % % %% % % % % % % % % % % % % % % % % % % %
% % % % % % % % % % % % % % % % % % % % % % % % % % %% % % % % % % % % % % % % % % % % % % %

\centerline{\sc Problem 1: Enzyme-based DNA Computing}
\vspace{0.2in}

Design a set of enzyme-based DNA logic gates (AND gate and OR gate) with emphasis on: 

$\bullet$ (a) leak reduction, explain how you aim to design your gates to be leakless \\
\textbf{Answer:} \\
\textbf{OR gate}:
An OR is a gate with two inputs and one output. It sets the output to “1“ when there is a
binary “1” in at least one of its inputs. Its true table is as follows:\\
  	\includegraphics{ORgate.png} \\
\centerline{\textit{An example of OR gate} }\\

\textbf{AND gate}:
An AND gate is a two input gate that only sets the output to “1“ when both inputs are “1“.
The true table of this device is: \\
  	\includegraphics{ANDgate.png} \\
\centerline{\textit{An example of AND gate} }\\
% introduce LEAK concept
\textbf{LEAK:} incorrect signal production.
\newline
\includegraphics{Andgatefigure.png} \\
\centerline{\textit{An implementation of X := AND(A, B) using double-long domain motif}} \\





\vspace{0.1in}
$\bullet$ (b) scalability which means that the gates should be able to work together as a circuit. 
You are allowed to use any enzyme.\\


Cite: 1 Design of enzyme-interfaced DNA logic operations (AND, OR and INHIBIT) with an assaying application for single-base mismatch \\
	  2 Leakless DNA Strand Displacement Systems Fig 4 and Fig 5
	  3 Genetic Logic Gates and Flipping DNA, \url{http://www.nature.com/scitable/blog/bio2.0/genetic_logic_gates_and_flipping}
	  4 A simple DNA gate motif for synthesizing large-scale circuits









% % % % % % % % % % % % % % % % % % % % % % % % % % %% % % % % % % % % % % % % % % % % % % %
% % % % % % % % % % % % % % % % % % % % % % % % % % %% % % % % % % % % % % % % % % % % % % %
% % % % % % % % % % % % % % % % % % % % % % % % % % %% % % % % % % % % % % % % % % % % % % %
\newpage
\centerline{\sc Problem 2: Enzyme-based DNA Computing}
\vspace{0.2in}
$\bullet$ (2a) Using class materials as well as literature search, find two different deoxyribozymes (DNAzymes) and describe in detail their properties. Using the DNAzymes you have chosen, construct a half-adder and a full-adder. Please cite all sources you have used to answer this question.\\

cite: \url{http://citeseerx.ist.psu.edu/viewdoc/download?doi=10.1.1.87.1221&rep=rep1&type=pdf}\\

\textbf{Answer:} \\
\textbf{Half-adder:} A half adder gives an output consisting of two bits. One of them is the binary addition of the two inputs, in other words, the result of a XOR operation between the two bits. The other
bit is the carry which can be implemented as the AND operation over the two inputs. \\
	\includegraphics{XORgate.png} \\
\centerline{\textit{An example of XOR gate} }\\

	
$\bullet$ (2b) Using class materials as well as literature search, construct hairpin-based OR, AND, and XOR gates. Describe an implementation of ripple-carry adder with the gates you have answered from the previous part. Please cite all sources you have used to answer this question. \\

\textbf{Answer:} \\
\textbf{Ripple Carry Adder:} A ripple carry adder is a digital circuit that produces the arithmetic sum of two binary numbers. It. can be constructed with full adders connected in cascaded (see section 2.1), with the carry output. from each full adder connected to the carry input of the next full adder in the chain. \\








% % % % % % % % % % % % % % % % % % % % % % % % % % %% % % % % % % % % % % % % % % % % % % %
% % % % % % % % % % % % % % % % % % % % % % % % % % %% % % % % % % % % % % % % % % % % % % %
% % % % % % % % % % % % % % % % % % % % % % % % % % %% % % % % % % % % % % % % % % % % % % %
\newpage
\centerline{\sc Problem 3:  Hybridization-based DNA Computing}
\vspace{0.2in}
$\bullet$ (3a) Read the article: \url{ http://science.sciencemag.org/content/332/6034/1196} and write a paragraph of summary.\\ 


$\bullet$ (3b) Design a DNA digital circuit to compute square roots for 2-bit numbers using seesaw gates. \\


$\bullet$ (3c) Simulate your circuit using LBS (Language for Biological Systems) in Visual GEC: \\

LBS (Language for Biological Systems): see  
	\url{ http://homepages.inf.ed.ac.uk/gdp/publications/Lang_Bio_Sys_Design_Spec.pdf }
	
	Visual GEC: see \url{ http://lepton.research.microsoft.com/webgec/ }





% % % % % % % % % % % % % % % % % % % % % % % % % % %% % % % % % % % % % % % % % % % % % % %
% % % % % % % % % % % % % % % % % % % % % % % % % % %% % % % % % % % % % % % % % % % % % % %
% % % % % % % % % % % % % % % % % % % % % % % % % % %% % % % % % % % % % % % % % % % % % % %
\newpage
\centerline{\sc Problem 4:   (DNA-based Molecular Robotics)}
\vspace{0.2in}
$\bullet$ (4a) Every robotics system requires fuel to function. List the different types of fuel used currently in DNA robotic reactions and mention at least 3 examples (with references) for each. \\

$\bullet$(4b) What role does concentration play in fueling robotic motion? Explain this for each of the different types of fuel.  \\

$\bullet$(4c) List 3 different ways of observing/reporting/confirming robotic motion (fluorescence, imaging, gel) and provide 3 examples (with references) on each.  \\

$\bullet$(4d) Summarize an example DNAzyne-based or hybridization based robotics demonstration from 2014 onwards, and argue why you consider it to be so.







% % % % % % % % % % % % % % % % % % % % % % % % % % %% % % % % % % % % % % % % % % % % % % %
% % % % % % % % % % % % % % % % % % % % % % % % % % %% % % % % % % % % % % % % % % % % % % %
% % % % % % % % % % % % % % % % % % % % % % % % % % %% % % % % % % % % % % % % % % % % % % %
\newpage
\centerline{\sc Problem 5:  Enzyme based DNA Robotics:}
\vspace{0.2in}
DNA Robotics is the study of designing DNA systems that can perform conformational changes/ display locomotion. In order to perform any change, energy is required. Let us take the example of a DNA walker - 
"A Unidirectional DNA Walker That Moves Autonomously along a Track"
(See \url{ https://users.cs.duke.edu/~reif/paper/harish/NanoRobotics/NanoRobotics.pdf } for an easier explanation of the article).


$\bullet$ (5a) What is the source of energy for this locomotion. Give the chemical reaction that is involved in this system that releases energy.\\

$\bullet$ (5b) Is this reaction reversible, i.e. can the walker move in the reverse direction?\\


$\bullet$ (5c) List all the enzymes involved in this particular walker, and the DNA recognition sequences that they cut. \\

$\bullet$ (5d) Currently this design is limited to 2 steps. Can this design be extended to 4 steps. If so, how? Will you need more enzymes?\\





% % % % % % % % % % % % % % % % % % % % % % % % % % %% % % % % % % % % % % % % % % % % % % %
% % % % % % % % % % % % % % % % % % % % % % % % % % %% % % % % % % % % % % % % % % % % % % %
% % % % % % % % % % % % % % % % % % % % % % % % % % %% % % % % % % % % % % % % % % % % % % %
\newpage
\centerline{\sc Problem 6:    DNAZyme based DNA Robotics:}
\vspace{0.2in}


$\bullet$ (6a) Read the article. Molecular robots guided by prescriptive landscapes. \\

$\bullet$ (6b) What is the difference between a DNAzyme and an enzyme. In one sentence. \\

$\bullet$ (6c) What is the source of energy for this locomotion. Give the chemical reaction that is involved in this system that releases energy. \\

$\bullet$ (6d) What type of molecule is cleaved by the DNAzyme?\\







% % % % % % % % % % % % % % % % % % % % % % % % % % %% % % % % % % % % % % % % % % % % % % %
% % % % % % % % % % % % % % % % % % % % % % % % % % %% % % % % % % % % % % % % % % % % % % %
% % % % % % % % % % % % % % % % % % % % % % % % % % %% % % % % % % % % % % % % % % % % % % %
\newpage
\centerline{\sc Problem 7:  DNA-based Molecular Robotics Design}
\vspace{0.2in}


$\bullet$ Design an arbitrary track, and a simple DNAzyme-based or hybridization-based DNA robotic system which can move along that track. You can use any sort of track. Submit the design to a simulation software (using NUPACK or canDo, for example, or both) with a simulation of two or more steps of locomotion, and provide the results of the simulation .




% % % % % % % % % % % % % % % % % % % % % % % % % % %% % % % % % % % % % % % % % % % % % % %
% % % % % % % % % % % % % % % % % % % % % % % % % % %% % % % % % % % % % % % % % % % % % % %
% % % % % % % % % % % % % % % % % % % % % % % % % % %% % % % % % % % % % % % % % % % % % % %
\end{document}